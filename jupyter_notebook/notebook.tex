
% Default to the notebook output style

    


% Inherit from the specified cell style.




    
\documentclass[11pt]{article}

    
    
    \usepackage[T1]{fontenc}
    % Nicer default font (+ math font) than Computer Modern for most use cases
    \usepackage{mathpazo}

    % Basic figure setup, for now with no caption control since it's done
    % automatically by Pandoc (which extracts ![](path) syntax from Markdown).
    \usepackage{graphicx}
    % We will generate all images so they have a width \maxwidth. This means
    % that they will get their normal width if they fit onto the page, but
    % are scaled down if they would overflow the margins.
    \makeatletter
    \def\maxwidth{\ifdim\Gin@nat@width>\linewidth\linewidth
    \else\Gin@nat@width\fi}
    \makeatother
    \let\Oldincludegraphics\includegraphics
    % Set max figure width to be 80% of text width, for now hardcoded.
    \renewcommand{\includegraphics}[1]{\Oldincludegraphics[width=.8\maxwidth]{#1}}
    % Ensure that by default, figures have no caption (until we provide a
    % proper Figure object with a Caption API and a way to capture that
    % in the conversion process - todo).
    \usepackage{caption}
    \DeclareCaptionLabelFormat{nolabel}{}
    \captionsetup{labelformat=nolabel}

    \usepackage{adjustbox} % Used to constrain images to a maximum size 
    \usepackage{xcolor} % Allow colors to be defined
    \usepackage{enumerate} % Needed for markdown enumerations to work
    \usepackage{geometry} % Used to adjust the document margins
    \usepackage{amsmath} % Equations
    \usepackage{amssymb} % Equations
    \usepackage{textcomp} % defines textquotesingle
    % Hack from http://tex.stackexchange.com/a/47451/13684:
    \AtBeginDocument{%
        \def\PYZsq{\textquotesingle}% Upright quotes in Pygmentized code
    }
    \usepackage{upquote} % Upright quotes for verbatim code
    \usepackage{eurosym} % defines \euro
    \usepackage[mathletters]{ucs} % Extended unicode (utf-8) support
    \usepackage[utf8x]{inputenc} % Allow utf-8 characters in the tex document
    \usepackage{fancyvrb} % verbatim replacement that allows latex
    \usepackage{grffile} % extends the file name processing of package graphics 
                         % to support a larger range 
    % The hyperref package gives us a pdf with properly built
    % internal navigation ('pdf bookmarks' for the table of contents,
    % internal cross-reference links, web links for URLs, etc.)
    \usepackage{hyperref}
    \usepackage{longtable} % longtable support required by pandoc >1.10
    \usepackage{booktabs}  % table support for pandoc > 1.12.2
    \usepackage[inline]{enumitem} % IRkernel/repr support (it uses the enumerate* environment)
    \usepackage[normalem]{ulem} % ulem is needed to support strikethroughs (\sout)
                                % normalem makes italics be italics, not underlines
    

    
    
    % Colors for the hyperref package
    \definecolor{urlcolor}{rgb}{0,.145,.698}
    \definecolor{linkcolor}{rgb}{.71,0.21,0.01}
    \definecolor{citecolor}{rgb}{.12,.54,.11}

    % ANSI colors
    \definecolor{ansi-black}{HTML}{3E424D}
    \definecolor{ansi-black-intense}{HTML}{282C36}
    \definecolor{ansi-red}{HTML}{E75C58}
    \definecolor{ansi-red-intense}{HTML}{B22B31}
    \definecolor{ansi-green}{HTML}{00A250}
    \definecolor{ansi-green-intense}{HTML}{007427}
    \definecolor{ansi-yellow}{HTML}{DDB62B}
    \definecolor{ansi-yellow-intense}{HTML}{B27D12}
    \definecolor{ansi-blue}{HTML}{208FFB}
    \definecolor{ansi-blue-intense}{HTML}{0065CA}
    \definecolor{ansi-magenta}{HTML}{D160C4}
    \definecolor{ansi-magenta-intense}{HTML}{A03196}
    \definecolor{ansi-cyan}{HTML}{60C6C8}
    \definecolor{ansi-cyan-intense}{HTML}{258F8F}
    \definecolor{ansi-white}{HTML}{C5C1B4}
    \definecolor{ansi-white-intense}{HTML}{A1A6B2}

    % commands and environments needed by pandoc snippets
    % extracted from the output of `pandoc -s`
    \providecommand{\tightlist}{%
      \setlength{\itemsep}{0pt}\setlength{\parskip}{0pt}}
    \DefineVerbatimEnvironment{Highlighting}{Verbatim}{commandchars=\\\{\}}
    % Add ',fontsize=\small' for more characters per line
    \newenvironment{Shaded}{}{}
    \newcommand{\KeywordTok}[1]{\textcolor[rgb]{0.00,0.44,0.13}{\textbf{{#1}}}}
    \newcommand{\DataTypeTok}[1]{\textcolor[rgb]{0.56,0.13,0.00}{{#1}}}
    \newcommand{\DecValTok}[1]{\textcolor[rgb]{0.25,0.63,0.44}{{#1}}}
    \newcommand{\BaseNTok}[1]{\textcolor[rgb]{0.25,0.63,0.44}{{#1}}}
    \newcommand{\FloatTok}[1]{\textcolor[rgb]{0.25,0.63,0.44}{{#1}}}
    \newcommand{\CharTok}[1]{\textcolor[rgb]{0.25,0.44,0.63}{{#1}}}
    \newcommand{\StringTok}[1]{\textcolor[rgb]{0.25,0.44,0.63}{{#1}}}
    \newcommand{\CommentTok}[1]{\textcolor[rgb]{0.38,0.63,0.69}{\textit{{#1}}}}
    \newcommand{\OtherTok}[1]{\textcolor[rgb]{0.00,0.44,0.13}{{#1}}}
    \newcommand{\AlertTok}[1]{\textcolor[rgb]{1.00,0.00,0.00}{\textbf{{#1}}}}
    \newcommand{\FunctionTok}[1]{\textcolor[rgb]{0.02,0.16,0.49}{{#1}}}
    \newcommand{\RegionMarkerTok}[1]{{#1}}
    \newcommand{\ErrorTok}[1]{\textcolor[rgb]{1.00,0.00,0.00}{\textbf{{#1}}}}
    \newcommand{\NormalTok}[1]{{#1}}
    
    % Additional commands for more recent versions of Pandoc
    \newcommand{\ConstantTok}[1]{\textcolor[rgb]{0.53,0.00,0.00}{{#1}}}
    \newcommand{\SpecialCharTok}[1]{\textcolor[rgb]{0.25,0.44,0.63}{{#1}}}
    \newcommand{\VerbatimStringTok}[1]{\textcolor[rgb]{0.25,0.44,0.63}{{#1}}}
    \newcommand{\SpecialStringTok}[1]{\textcolor[rgb]{0.73,0.40,0.53}{{#1}}}
    \newcommand{\ImportTok}[1]{{#1}}
    \newcommand{\DocumentationTok}[1]{\textcolor[rgb]{0.73,0.13,0.13}{\textit{{#1}}}}
    \newcommand{\AnnotationTok}[1]{\textcolor[rgb]{0.38,0.63,0.69}{\textbf{\textit{{#1}}}}}
    \newcommand{\CommentVarTok}[1]{\textcolor[rgb]{0.38,0.63,0.69}{\textbf{\textit{{#1}}}}}
    \newcommand{\VariableTok}[1]{\textcolor[rgb]{0.10,0.09,0.49}{{#1}}}
    \newcommand{\ControlFlowTok}[1]{\textcolor[rgb]{0.00,0.44,0.13}{\textbf{{#1}}}}
    \newcommand{\OperatorTok}[1]{\textcolor[rgb]{0.40,0.40,0.40}{{#1}}}
    \newcommand{\BuiltInTok}[1]{{#1}}
    \newcommand{\ExtensionTok}[1]{{#1}}
    \newcommand{\PreprocessorTok}[1]{\textcolor[rgb]{0.74,0.48,0.00}{{#1}}}
    \newcommand{\AttributeTok}[1]{\textcolor[rgb]{0.49,0.56,0.16}{{#1}}}
    \newcommand{\InformationTok}[1]{\textcolor[rgb]{0.38,0.63,0.69}{\textbf{\textit{{#1}}}}}
    \newcommand{\WarningTok}[1]{\textcolor[rgb]{0.38,0.63,0.69}{\textbf{\textit{{#1}}}}}
    
    
    % Define a nice break command that doesn't care if a line doesn't already
    % exist.
    \def\br{\hspace*{\fill} \\* }
    % Math Jax compatability definitions
    \def\gt{>}
    \def\lt{<}
    % Document parameters
    \title{Auswertung}
    
    
    

    % Pygments definitions
    
\makeatletter
\def\PY@reset{\let\PY@it=\relax \let\PY@bf=\relax%
    \let\PY@ul=\relax \let\PY@tc=\relax%
    \let\PY@bc=\relax \let\PY@ff=\relax}
\def\PY@tok#1{\csname PY@tok@#1\endcsname}
\def\PY@toks#1+{\ifx\relax#1\empty\else%
    \PY@tok{#1}\expandafter\PY@toks\fi}
\def\PY@do#1{\PY@bc{\PY@tc{\PY@ul{%
    \PY@it{\PY@bf{\PY@ff{#1}}}}}}}
\def\PY#1#2{\PY@reset\PY@toks#1+\relax+\PY@do{#2}}

\expandafter\def\csname PY@tok@w\endcsname{\def\PY@tc##1{\textcolor[rgb]{0.73,0.73,0.73}{##1}}}
\expandafter\def\csname PY@tok@c\endcsname{\let\PY@it=\textit\def\PY@tc##1{\textcolor[rgb]{0.25,0.50,0.50}{##1}}}
\expandafter\def\csname PY@tok@cp\endcsname{\def\PY@tc##1{\textcolor[rgb]{0.74,0.48,0.00}{##1}}}
\expandafter\def\csname PY@tok@k\endcsname{\let\PY@bf=\textbf\def\PY@tc##1{\textcolor[rgb]{0.00,0.50,0.00}{##1}}}
\expandafter\def\csname PY@tok@kp\endcsname{\def\PY@tc##1{\textcolor[rgb]{0.00,0.50,0.00}{##1}}}
\expandafter\def\csname PY@tok@kt\endcsname{\def\PY@tc##1{\textcolor[rgb]{0.69,0.00,0.25}{##1}}}
\expandafter\def\csname PY@tok@o\endcsname{\def\PY@tc##1{\textcolor[rgb]{0.40,0.40,0.40}{##1}}}
\expandafter\def\csname PY@tok@ow\endcsname{\let\PY@bf=\textbf\def\PY@tc##1{\textcolor[rgb]{0.67,0.13,1.00}{##1}}}
\expandafter\def\csname PY@tok@nb\endcsname{\def\PY@tc##1{\textcolor[rgb]{0.00,0.50,0.00}{##1}}}
\expandafter\def\csname PY@tok@nf\endcsname{\def\PY@tc##1{\textcolor[rgb]{0.00,0.00,1.00}{##1}}}
\expandafter\def\csname PY@tok@nc\endcsname{\let\PY@bf=\textbf\def\PY@tc##1{\textcolor[rgb]{0.00,0.00,1.00}{##1}}}
\expandafter\def\csname PY@tok@nn\endcsname{\let\PY@bf=\textbf\def\PY@tc##1{\textcolor[rgb]{0.00,0.00,1.00}{##1}}}
\expandafter\def\csname PY@tok@ne\endcsname{\let\PY@bf=\textbf\def\PY@tc##1{\textcolor[rgb]{0.82,0.25,0.23}{##1}}}
\expandafter\def\csname PY@tok@nv\endcsname{\def\PY@tc##1{\textcolor[rgb]{0.10,0.09,0.49}{##1}}}
\expandafter\def\csname PY@tok@no\endcsname{\def\PY@tc##1{\textcolor[rgb]{0.53,0.00,0.00}{##1}}}
\expandafter\def\csname PY@tok@nl\endcsname{\def\PY@tc##1{\textcolor[rgb]{0.63,0.63,0.00}{##1}}}
\expandafter\def\csname PY@tok@ni\endcsname{\let\PY@bf=\textbf\def\PY@tc##1{\textcolor[rgb]{0.60,0.60,0.60}{##1}}}
\expandafter\def\csname PY@tok@na\endcsname{\def\PY@tc##1{\textcolor[rgb]{0.49,0.56,0.16}{##1}}}
\expandafter\def\csname PY@tok@nt\endcsname{\let\PY@bf=\textbf\def\PY@tc##1{\textcolor[rgb]{0.00,0.50,0.00}{##1}}}
\expandafter\def\csname PY@tok@nd\endcsname{\def\PY@tc##1{\textcolor[rgb]{0.67,0.13,1.00}{##1}}}
\expandafter\def\csname PY@tok@s\endcsname{\def\PY@tc##1{\textcolor[rgb]{0.73,0.13,0.13}{##1}}}
\expandafter\def\csname PY@tok@sd\endcsname{\let\PY@it=\textit\def\PY@tc##1{\textcolor[rgb]{0.73,0.13,0.13}{##1}}}
\expandafter\def\csname PY@tok@si\endcsname{\let\PY@bf=\textbf\def\PY@tc##1{\textcolor[rgb]{0.73,0.40,0.53}{##1}}}
\expandafter\def\csname PY@tok@se\endcsname{\let\PY@bf=\textbf\def\PY@tc##1{\textcolor[rgb]{0.73,0.40,0.13}{##1}}}
\expandafter\def\csname PY@tok@sr\endcsname{\def\PY@tc##1{\textcolor[rgb]{0.73,0.40,0.53}{##1}}}
\expandafter\def\csname PY@tok@ss\endcsname{\def\PY@tc##1{\textcolor[rgb]{0.10,0.09,0.49}{##1}}}
\expandafter\def\csname PY@tok@sx\endcsname{\def\PY@tc##1{\textcolor[rgb]{0.00,0.50,0.00}{##1}}}
\expandafter\def\csname PY@tok@m\endcsname{\def\PY@tc##1{\textcolor[rgb]{0.40,0.40,0.40}{##1}}}
\expandafter\def\csname PY@tok@gh\endcsname{\let\PY@bf=\textbf\def\PY@tc##1{\textcolor[rgb]{0.00,0.00,0.50}{##1}}}
\expandafter\def\csname PY@tok@gu\endcsname{\let\PY@bf=\textbf\def\PY@tc##1{\textcolor[rgb]{0.50,0.00,0.50}{##1}}}
\expandafter\def\csname PY@tok@gd\endcsname{\def\PY@tc##1{\textcolor[rgb]{0.63,0.00,0.00}{##1}}}
\expandafter\def\csname PY@tok@gi\endcsname{\def\PY@tc##1{\textcolor[rgb]{0.00,0.63,0.00}{##1}}}
\expandafter\def\csname PY@tok@gr\endcsname{\def\PY@tc##1{\textcolor[rgb]{1.00,0.00,0.00}{##1}}}
\expandafter\def\csname PY@tok@ge\endcsname{\let\PY@it=\textit}
\expandafter\def\csname PY@tok@gs\endcsname{\let\PY@bf=\textbf}
\expandafter\def\csname PY@tok@gp\endcsname{\let\PY@bf=\textbf\def\PY@tc##1{\textcolor[rgb]{0.00,0.00,0.50}{##1}}}
\expandafter\def\csname PY@tok@go\endcsname{\def\PY@tc##1{\textcolor[rgb]{0.53,0.53,0.53}{##1}}}
\expandafter\def\csname PY@tok@gt\endcsname{\def\PY@tc##1{\textcolor[rgb]{0.00,0.27,0.87}{##1}}}
\expandafter\def\csname PY@tok@err\endcsname{\def\PY@bc##1{\setlength{\fboxsep}{0pt}\fcolorbox[rgb]{1.00,0.00,0.00}{1,1,1}{\strut ##1}}}
\expandafter\def\csname PY@tok@kc\endcsname{\let\PY@bf=\textbf\def\PY@tc##1{\textcolor[rgb]{0.00,0.50,0.00}{##1}}}
\expandafter\def\csname PY@tok@kd\endcsname{\let\PY@bf=\textbf\def\PY@tc##1{\textcolor[rgb]{0.00,0.50,0.00}{##1}}}
\expandafter\def\csname PY@tok@kn\endcsname{\let\PY@bf=\textbf\def\PY@tc##1{\textcolor[rgb]{0.00,0.50,0.00}{##1}}}
\expandafter\def\csname PY@tok@kr\endcsname{\let\PY@bf=\textbf\def\PY@tc##1{\textcolor[rgb]{0.00,0.50,0.00}{##1}}}
\expandafter\def\csname PY@tok@bp\endcsname{\def\PY@tc##1{\textcolor[rgb]{0.00,0.50,0.00}{##1}}}
\expandafter\def\csname PY@tok@fm\endcsname{\def\PY@tc##1{\textcolor[rgb]{0.00,0.00,1.00}{##1}}}
\expandafter\def\csname PY@tok@vc\endcsname{\def\PY@tc##1{\textcolor[rgb]{0.10,0.09,0.49}{##1}}}
\expandafter\def\csname PY@tok@vg\endcsname{\def\PY@tc##1{\textcolor[rgb]{0.10,0.09,0.49}{##1}}}
\expandafter\def\csname PY@tok@vi\endcsname{\def\PY@tc##1{\textcolor[rgb]{0.10,0.09,0.49}{##1}}}
\expandafter\def\csname PY@tok@vm\endcsname{\def\PY@tc##1{\textcolor[rgb]{0.10,0.09,0.49}{##1}}}
\expandafter\def\csname PY@tok@sa\endcsname{\def\PY@tc##1{\textcolor[rgb]{0.73,0.13,0.13}{##1}}}
\expandafter\def\csname PY@tok@sb\endcsname{\def\PY@tc##1{\textcolor[rgb]{0.73,0.13,0.13}{##1}}}
\expandafter\def\csname PY@tok@sc\endcsname{\def\PY@tc##1{\textcolor[rgb]{0.73,0.13,0.13}{##1}}}
\expandafter\def\csname PY@tok@dl\endcsname{\def\PY@tc##1{\textcolor[rgb]{0.73,0.13,0.13}{##1}}}
\expandafter\def\csname PY@tok@s2\endcsname{\def\PY@tc##1{\textcolor[rgb]{0.73,0.13,0.13}{##1}}}
\expandafter\def\csname PY@tok@sh\endcsname{\def\PY@tc##1{\textcolor[rgb]{0.73,0.13,0.13}{##1}}}
\expandafter\def\csname PY@tok@s1\endcsname{\def\PY@tc##1{\textcolor[rgb]{0.73,0.13,0.13}{##1}}}
\expandafter\def\csname PY@tok@mb\endcsname{\def\PY@tc##1{\textcolor[rgb]{0.40,0.40,0.40}{##1}}}
\expandafter\def\csname PY@tok@mf\endcsname{\def\PY@tc##1{\textcolor[rgb]{0.40,0.40,0.40}{##1}}}
\expandafter\def\csname PY@tok@mh\endcsname{\def\PY@tc##1{\textcolor[rgb]{0.40,0.40,0.40}{##1}}}
\expandafter\def\csname PY@tok@mi\endcsname{\def\PY@tc##1{\textcolor[rgb]{0.40,0.40,0.40}{##1}}}
\expandafter\def\csname PY@tok@il\endcsname{\def\PY@tc##1{\textcolor[rgb]{0.40,0.40,0.40}{##1}}}
\expandafter\def\csname PY@tok@mo\endcsname{\def\PY@tc##1{\textcolor[rgb]{0.40,0.40,0.40}{##1}}}
\expandafter\def\csname PY@tok@ch\endcsname{\let\PY@it=\textit\def\PY@tc##1{\textcolor[rgb]{0.25,0.50,0.50}{##1}}}
\expandafter\def\csname PY@tok@cm\endcsname{\let\PY@it=\textit\def\PY@tc##1{\textcolor[rgb]{0.25,0.50,0.50}{##1}}}
\expandafter\def\csname PY@tok@cpf\endcsname{\let\PY@it=\textit\def\PY@tc##1{\textcolor[rgb]{0.25,0.50,0.50}{##1}}}
\expandafter\def\csname PY@tok@c1\endcsname{\let\PY@it=\textit\def\PY@tc##1{\textcolor[rgb]{0.25,0.50,0.50}{##1}}}
\expandafter\def\csname PY@tok@cs\endcsname{\let\PY@it=\textit\def\PY@tc##1{\textcolor[rgb]{0.25,0.50,0.50}{##1}}}

\def\PYZbs{\char`\\}
\def\PYZus{\char`\_}
\def\PYZob{\char`\{}
\def\PYZcb{\char`\}}
\def\PYZca{\char`\^}
\def\PYZam{\char`\&}
\def\PYZlt{\char`\<}
\def\PYZgt{\char`\>}
\def\PYZsh{\char`\#}
\def\PYZpc{\char`\%}
\def\PYZdl{\char`\$}
\def\PYZhy{\char`\-}
\def\PYZsq{\char`\'}
\def\PYZdq{\char`\"}
\def\PYZti{\char`\~}
% for compatibility with earlier versions
\def\PYZat{@}
\def\PYZlb{[}
\def\PYZrb{]}
\makeatother


    % Exact colors from NB
    \definecolor{incolor}{rgb}{0.0, 0.0, 0.5}
    \definecolor{outcolor}{rgb}{0.545, 0.0, 0.0}



    
    % Prevent overflowing lines due to hard-to-break entities
    \sloppy 
    % Setup hyperref package
    \hypersetup{
      breaklinks=true,  % so long urls are correctly broken across lines
      colorlinks=true,
      urlcolor=urlcolor,
      linkcolor=linkcolor,
      citecolor=citecolor,
      }
    % Slightly bigger margins than the latex defaults
    
    \geometry{verbose,tmargin=1in,bmargin=1in,lmargin=1in,rmargin=1in}
    
    

    \begin{document}
    
    
    \maketitle
    
    

    
    \section{Einbinden der Bibliotheken}\label{einbinden-der-bibliotheken}

    \begin{Verbatim}[commandchars=\\\{\}]
{\color{incolor}In [{\color{incolor}12}]:} \PY{c+c1}{\PYZsh{} Imports}
         \PY{k+kn}{import} \PY{n+nn}{os}
         \PY{k+kn}{import} \PY{n+nn}{json}
         
         \PY{k+kn}{from} \PY{n+nn}{scipy}\PY{n+nn}{.}\PY{n+nn}{interpolate} \PY{k}{import} \PY{n}{interp1d}
         \PY{k+kn}{from} \PY{n+nn}{scipy}\PY{n+nn}{.}\PY{n+nn}{signal} \PY{k}{import} \PY{n}{butter}\PY{p}{,} \PY{n}{lfilter}
         \PY{k+kn}{import} \PY{n+nn}{numpy} \PY{k}{as} \PY{n+nn}{np}
         \PY{k+kn}{from} \PY{n+nn}{matplotlib} \PY{k}{import} \PY{n}{pyplot} \PY{k}{as} \PY{n}{plt}
         
         \PY{c+c1}{\PYZsh{} Configure plot size}
         \PY{n}{plt}\PY{o}{.}\PY{n}{rcParams}\PY{p}{[}\PY{l+s+s2}{\PYZdq{}}\PY{l+s+s2}{figure.figsize}\PY{l+s+s2}{\PYZdq{}}\PY{p}{]} \PY{o}{=} \PY{p}{(}\PY{l+m+mi}{10}\PY{p}{,}\PY{l+m+mi}{5}\PY{p}{)}
\end{Verbatim}


    \section{Laden der Datei}\label{laden-der-datei}

Während der Messung werden die empfangenen Daten in Listen gespeichert.
Diese werden nach Beendung der Messung als JSON-Datei gespeichert. Um
das aufgezeichnete Signal auszuwerten, wird dieses zunächst geladen. Die
einzelnen Messwerte und die dazugehörenden Zeitwerte werden in zwei
Listen (x, y) gespeichert.

    \begin{Verbatim}[commandchars=\\\{\}]
{\color{incolor}In [{\color{incolor}13}]:} \PY{n}{DIRECTORY} \PY{o}{=} \PY{l+s+s1}{\PYZsq{}}\PY{l+s+s1}{../records}\PY{l+s+s1}{\PYZsq{}}
         
         \PY{k}{def} \PY{n+nf}{get\PYZus{}signal}\PY{p}{(}\PY{n}{filename}\PY{p}{)}\PY{p}{:}
             \PY{k}{with} \PY{n+nb}{open}\PY{p}{(}\PY{n}{os}\PY{o}{.}\PY{n}{path}\PY{o}{.}\PY{n}{join}\PY{p}{(}\PY{n}{DIRECTORY}\PY{p}{,} \PY{n}{filename}\PY{p}{)}\PY{p}{)} \PY{k}{as} \PY{n}{file}\PY{p}{:}
                 \PY{k}{return} \PY{n}{json}\PY{o}{.}\PY{n}{load}\PY{p}{(}\PY{n}{file}\PY{p}{)}
             
         \PY{n}{signal} \PY{o}{=} \PY{n}{get\PYZus{}signal}\PY{p}{(}\PY{l+s+s1}{\PYZsq{}}\PY{l+s+s1}{20180429\PYZhy{}212736\PYZus{}Maria\PYZus{}gd\PYZus{}.json}\PY{l+s+s1}{\PYZsq{}}\PY{p}{)}
         \PY{n}{x} \PY{o}{=} \PY{n}{signal}\PY{p}{[}\PY{l+s+s1}{\PYZsq{}}\PY{l+s+s1}{times}\PY{l+s+s1}{\PYZsq{}}\PY{p}{]}
         \PY{n}{y} \PY{o}{=} \PY{n}{signal}\PY{p}{[}\PY{l+s+s1}{\PYZsq{}}\PY{l+s+s1}{channels}\PY{l+s+s1}{\PYZsq{}}\PY{p}{]}\PY{p}{[}\PY{l+m+mi}{0}\PY{p}{]}
\end{Verbatim}


    \section{Verarbeitung der Rohdaten}\label{verarbeitung-der-rohdaten}

Das empfangene, ungefilterte Signal ist zunächst unbrauchbar. Es
beinhaltet sehr viele Störfrequenzen.

    \subsection{Resampling}\label{resampling}

Wie man beispielhaft aus der Zeitdifferenz der ersten 10 Messwerte
erkennen kann, ist die Abtastrate nicht konstant.

    \begin{Verbatim}[commandchars=\\\{\}]
{\color{incolor}In [{\color{incolor}14}]:} \PY{p}{[}\PY{n}{x}\PY{p}{[}\PY{n}{i}\PY{o}{+}\PY{l+m+mi}{1}\PY{p}{]} \PY{o}{\PYZhy{}} \PY{n}{x}\PY{p}{[}\PY{n}{i}\PY{p}{]} \PY{k}{for} \PY{n}{i} \PY{o+ow}{in} \PY{n+nb}{range}\PY{p}{(}\PY{l+m+mi}{10}\PY{p}{)}\PY{p}{]}
\end{Verbatim}


\begin{Verbatim}[commandchars=\\\{\}]
{\color{outcolor}Out[{\color{outcolor}14}]:} [0.0005740560009144247,
          1.1071002518292516e-05,
          0.000546176997886505,
          1.0957999620586634e-05,
          0.0007707860058872029,
          1.3389995729085058e-05,
          0.04689133800275158,
          1.89829952432774e-05,
          0.0006394459996954538,
          1.233000512002036e-05]
\end{Verbatim}
            
    Die Inkonstanz wird in erster Linie durch das Buffering und somit
inkonstanten Verzögerungen bei der Übertragung oder durch Paketverluste
bei der Übertragung verursacht. Die meisten Algorithmen zur
Signalverarbeitung, wie beispielsweise Passfilter, benötigen jedoch ein
Signal mit konstanter Abtastrate. Das Signal wird deshalb zunächst auf
eine fixe Abtastrate von 100 Hz resampled.

    \begin{Verbatim}[commandchars=\\\{\}]
{\color{incolor}In [{\color{incolor}15}]:} \PY{k}{def} \PY{n+nf}{resample}\PY{p}{(}\PY{n}{x}\PY{p}{,} \PY{n}{y}\PY{p}{,} \PY{n}{frequency}\PY{p}{)}\PY{p}{:}
             \PY{c+c1}{\PYZsh{} Calculate the new sample amount and the period length of one sample}
             \PY{n}{length} \PY{o}{=} \PY{n}{x}\PY{p}{[}\PY{o}{\PYZhy{}}\PY{l+m+mi}{1}\PY{p}{]}
             \PY{n}{samples\PYZus{}amount} \PY{o}{=} \PY{n+nb}{int}\PY{p}{(}\PY{n}{length} \PY{o}{*} \PY{n}{frequency}\PY{p}{)}
             \PY{n}{period\PYZus{}length} \PY{o}{=} \PY{l+m+mi}{1} \PY{o}{/} \PY{n}{frequency}
             \PY{c+c1}{\PYZsh{} Create list with x\PYZhy{}values (time values) for the given frequency}
             \PY{c+c1}{\PYZsh{} First value has to be greater or equal the first value of the input}
             \PY{n}{x\PYZus{}new} \PY{o}{=} \PY{p}{[}\PY{n}{period\PYZus{}length} \PY{o}{*} \PY{n}{i} \PY{k}{for} \PY{n}{i} \PY{o+ow}{in} \PY{n+nb}{range}\PY{p}{(}\PY{n}{samples\PYZus{}amount}\PY{p}{)} \PY{k}{if} \PY{n}{period\PYZus{}length} \PY{o}{*} \PY{n}{i} \PY{o}{\PYZgt{}}\PY{o}{=} \PY{n}{x}\PY{p}{[}\PY{l+m+mi}{0}\PY{p}{]}\PY{p}{]}
             
             \PY{c+c1}{\PYZsh{} Calculate interpolated y values}
             \PY{n}{y\PYZus{}new} \PY{o}{=} \PY{p}{[}\PY{n}{interp1d}\PY{p}{(}\PY{n}{x}\PY{p}{,} \PY{n}{y}\PY{p}{)}\PY{p}{(}\PY{n}{i}\PY{p}{)}\PY{o}{.}\PY{n}{tolist}\PY{p}{(}\PY{p}{)} \PY{k}{for} \PY{n}{i} \PY{o+ow}{in} \PY{n}{x\PYZus{}new}\PY{p}{]}
         
             \PY{k}{return} \PY{n}{x\PYZus{}new}\PY{p}{,} \PY{n}{y\PYZus{}new}
         
         \PY{n}{FREQUENCY} \PY{o}{=} \PY{l+m+mi}{100} \PY{c+c1}{\PYZsh{} Hz}
         \PY{n}{x}\PY{p}{,} \PY{n}{y} \PY{o}{=} \PY{n}{resample}\PY{p}{(}\PY{n}{x}\PY{p}{,} \PY{n}{y}\PY{p}{,} \PY{n}{FREQUENCY}\PY{p}{)}
\end{Verbatim}


    \subsection{Filterung}\label{filterung}

\subsubsection{Schnelle
Fourier-Transformation}\label{schnelle-fourier-transformation}

Aus dem Graphen der Fourier-Transformation kann man erkennen, dass das
Signal ein breites Spektrum an Frequenzen enthält.

    \begin{Verbatim}[commandchars=\\\{\}]
{\color{incolor}In [{\color{incolor}28}]:} \PY{k}{def} \PY{n+nf}{fft}\PY{p}{(}\PY{n}{y}\PY{p}{,} \PY{n}{frequency}\PY{p}{)}\PY{p}{:}
             \PY{c+c1}{\PYZsh{} Get real amplitudes of FFT (only in positive frequencies)}
             \PY{n}{fft\PYZus{}values} \PY{o}{=} \PY{n}{np}\PY{o}{.}\PY{n}{absolute}\PY{p}{(}\PY{n}{np}\PY{o}{.}\PY{n}{fft}\PY{o}{.}\PY{n}{rfft}\PY{p}{(}\PY{n}{y}\PY{p}{)}\PY{p}{)}
         
             \PY{c+c1}{\PYZsh{} Get frequencies for amplitudes in Hz}
             \PY{n}{dt} \PY{o}{=} \PY{l+m+mi}{1} \PY{o}{/} \PY{n}{frequency}
             \PY{n}{fft\PYZus{}freq} \PY{o}{=} \PY{n}{np}\PY{o}{.}\PY{n}{fft}\PY{o}{.}\PY{n}{rfftfreq}\PY{p}{(}\PY{n+nb}{len}\PY{p}{(}\PY{n}{y}\PY{p}{)}\PY{p}{,} \PY{n}{dt}\PY{p}{)}
         
             \PY{k}{return} \PY{n}{fft\PYZus{}freq}\PY{p}{,} \PY{n}{fft\PYZus{}values}
         
         \PY{c+c1}{\PYZsh{} Plot Graph}
         \PY{n}{plt}\PY{o}{.}\PY{n}{plot}\PY{p}{(}\PY{o}{*}\PY{n}{fft}\PY{p}{(}\PY{n}{y}\PY{p}{,} \PY{n}{FREQUENCY}\PY{p}{)}\PY{p}{)}
         \PY{n}{plt}\PY{o}{.}\PY{n}{xlim}\PY{p}{(}\PY{l+m+mi}{0}\PY{p}{,}\PY{l+m+mi}{50}\PY{p}{)}
         \PY{n}{plt}\PY{o}{.}\PY{n}{ylim}\PY{p}{(}\PY{o}{\PYZhy{}}\PY{l+m+mi}{1}\PY{p}{,} \PY{l+m+mi}{20}\PY{p}{)}
         \PY{n}{plt}\PY{o}{.}\PY{n}{title}\PY{p}{(}\PY{l+s+s1}{\PYZsq{}}\PY{l+s+s1}{Fast Fourier transform}\PY{l+s+s1}{\PYZsq{}}\PY{p}{)}
         \PY{n}{plt}\PY{o}{.}\PY{n}{xlabel}\PY{p}{(}\PY{l+s+s1}{\PYZsq{}}\PY{l+s+s1}{f (Hz)}\PY{l+s+s1}{\PYZsq{}}\PY{p}{)}
         \PY{n}{plt}\PY{o}{.}\PY{n}{ylabel}\PY{p}{(}\PY{l+s+s1}{\PYZsq{}}\PY{l+s+s1}{P (μW)}\PY{l+s+s1}{\PYZsq{}}\PY{p}{)}
         \PY{n}{plt}\PY{o}{.}\PY{n}{show}\PY{p}{(}\PY{p}{)}
\end{Verbatim}


    \begin{center}
    \adjustimage{max size={0.9\linewidth}{0.9\paperheight}}{output_10_0.png}
    \end{center}
    { \hspace*{\fill} \\}
    
    \subsubsection{Bandpass-Filter}\label{bandpass-filter}

Man kann deutlich erkennen, dass sich die größte Leistung im Bereich
unter 20 Hz befindet. Gerade im hochfrequenten Bereich finden sich viele
Störsignale. Diese stammen hauptsächlich von drei Quellen: erstens von
körpereigenen Stromflüssen, welche nicht für die Messung relevant sind;
zweitens von externen Interferenzen (z.B.: Wechselstrom: 50 Hz) und
drittens werden sie einfach durch eine praktisch unmögliche 100\% exakte
Messung hervorgerufen. Um ein sauberes Signal zu erhalten werden diese
Frequenzen mittels eines Hochpass-Filters entfernd. Die richtige Wahl
der Grenzfrequenz ist dabei kritisch. Einerseits sollte sie niedrig
gehalten werden, um einen möglichst großen Bereich von Störfrequenzen zu
entfernen. Andererseits darf sie auch nicht zu klein sein, da dadurch
das Signal verfälscht wird. Da bei dieser Messung die Testperson
deutlich und nicht besonders schnell blinzelte, kann eine relativ kleine
Grenzfrequenz von 5 Hz für den Hochpassfilter verwendet werden.
Zusätzlich kann ein Tiefpass-Filter verwendet werden, welcher tiefe
Frequnzen unterhalb einer bestimmten Grenzfrequenz entfernd. Somit wird
das Signal unterschiedlich stark zu 0 V hingezogen. Bei der
Elektrookulografie ist dies grundsätzlich nicht sinnvoll. Der
eigentliche Messwert, aus dem man den Blickrichtungswinkel ableiten
kann, wird dadurch verfälscht. Bei der Detektierung des Blinzelns
hingegend ist nicht der exakte Wert der gemessenen Spannung zu einem
bestimmten Zeitpunk ausschlaggebend, sondern die Änderung der Spannung
innerhalb eines bestimmten Frequenzbereichs. Deshalb ist ein
Tiefpassfilter sinvoll, da dadurch die Peaks deutlicher erkennbar sind.
Hier wird eine Grenzfrequenz von 2 Hz für den Tiefpass-Filter verwendet.
In Summe bedeutet dies, dass nur der Frequenzbereich zwischen 2 und 5 Hz
betrachtet wird.

    \begin{Verbatim}[commandchars=\\\{\}]
{\color{incolor}In [{\color{incolor}17}]:} \PY{k}{def} \PY{n+nf}{butter\PYZus{}bandpass\PYZus{}filter}\PY{p}{(}\PY{n}{data}\PY{p}{,} \PY{n}{lowcut}\PY{p}{,} \PY{n}{highcut}\PY{p}{,} \PY{n}{frequency}\PY{p}{,} \PY{n}{order}\PY{o}{=}\PY{l+m+mi}{1}\PY{p}{)}\PY{p}{:}
             \PY{n}{nyq} \PY{o}{=} \PY{l+m+mf}{0.5} \PY{o}{*} \PY{n}{frequency}
             \PY{n}{low} \PY{o}{=} \PY{n}{lowcut} \PY{o}{/} \PY{n}{nyq}
             \PY{n}{high} \PY{o}{=} \PY{n}{highcut} \PY{o}{/} \PY{n}{nyq}
         
             \PY{n}{b}\PY{p}{,} \PY{n}{a} \PY{o}{=} \PY{n}{butter}\PY{p}{(}\PY{n}{order}\PY{p}{,} \PY{p}{[}\PY{n}{low}\PY{p}{,} \PY{n}{high}\PY{p}{]}\PY{p}{,} \PY{n}{btype}\PY{o}{=}\PY{l+s+s1}{\PYZsq{}}\PY{l+s+s1}{band}\PY{l+s+s1}{\PYZsq{}}\PY{p}{)}
             \PY{k}{return} \PY{n}{lfilter}\PY{p}{(}\PY{n}{b}\PY{p}{,} \PY{n}{a}\PY{p}{,} \PY{n}{data}\PY{p}{)}
         
         \PY{n}{LOWCUT} \PY{o}{=} \PY{l+m+mi}{2}
         \PY{n}{HIGHCUT} \PY{o}{=} \PY{l+m+mi}{5}
         \PY{n}{y\PYZus{}filtered} \PY{o}{=} \PY{n}{butter\PYZus{}bandpass\PYZus{}filter}\PY{p}{(}\PY{n}{y}\PY{p}{,} \PY{n}{LOWCUT}\PY{p}{,} \PY{n}{HIGHCUT}\PY{p}{,} \PY{n}{FREQUENCY}\PY{p}{)}
\end{Verbatim}


    Im folgendem Diagramm kann man die Graphen des ungefilterten Signals
(rot) und des gefilterten (blau) vergleichen. Man kann ausserdem
Beobachten, welche Auswirkung die Veränderung der Grenzfrequenzen des
Tief- und Hochpass-Filters auf das Signal hat.

    \begin{Verbatim}[commandchars=\\\{\}]
{\color{incolor}In [{\color{incolor}29}]:} \PY{n}{plt}\PY{o}{.}\PY{n}{plot}\PY{p}{(}\PY{n}{x}\PY{p}{,} \PY{n}{y}\PY{p}{,} \PY{n}{color}\PY{o}{=}\PY{l+s+s1}{\PYZsq{}}\PY{l+s+s1}{red}\PY{l+s+s1}{\PYZsq{}}\PY{p}{)}
         \PY{n}{plt}\PY{o}{.}\PY{n}{title}\PY{p}{(}\PY{l+s+s1}{\PYZsq{}}\PY{l+s+s1}{Unfiltered signal}\PY{l+s+s1}{\PYZsq{}}\PY{p}{)}
         \PY{n}{plt}\PY{o}{.}\PY{n}{xlabel}\PY{p}{(}\PY{l+s+s1}{\PYZsq{}}\PY{l+s+s1}{t (s)}\PY{l+s+s1}{\PYZsq{}}\PY{p}{)}
         \PY{n}{plt}\PY{o}{.}\PY{n}{ylabel}\PY{p}{(}\PY{l+s+s1}{\PYZsq{}}\PY{l+s+s1}{U (μV)}\PY{l+s+s1}{\PYZsq{}}\PY{p}{)}
         \PY{n}{plt}\PY{o}{.}\PY{n}{show}\PY{p}{(}\PY{p}{)}
         \PY{n}{plt}\PY{o}{.}\PY{n}{plot}\PY{p}{(}\PY{n}{x}\PY{p}{,} \PY{n}{y\PYZus{}filtered}\PY{p}{,} \PY{n}{color}\PY{o}{=}\PY{l+s+s1}{\PYZsq{}}\PY{l+s+s1}{blue}\PY{l+s+s1}{\PYZsq{}}\PY{p}{)}
         \PY{n}{plt}\PY{o}{.}\PY{n}{title}\PY{p}{(}\PY{l+s+s1}{\PYZsq{}}\PY{l+s+s1}{Filtered signal}\PY{l+s+s1}{\PYZsq{}}\PY{p}{)}
         \PY{n}{plt}\PY{o}{.}\PY{n}{xlabel}\PY{p}{(}\PY{l+s+s1}{\PYZsq{}}\PY{l+s+s1}{t (s)}\PY{l+s+s1}{\PYZsq{}}\PY{p}{)}
         \PY{n}{plt}\PY{o}{.}\PY{n}{ylabel}\PY{p}{(}\PY{l+s+s1}{\PYZsq{}}\PY{l+s+s1}{U (μV)}\PY{l+s+s1}{\PYZsq{}}\PY{p}{)}
         \PY{n}{plt}\PY{o}{.}\PY{n}{show}\PY{p}{(}\PY{p}{)}
\end{Verbatim}


    \begin{center}
    \adjustimage{max size={0.9\linewidth}{0.9\paperheight}}{output_14_0.png}
    \end{center}
    { \hspace*{\fill} \\}
    
    \begin{center}
    \adjustimage{max size={0.9\linewidth}{0.9\paperheight}}{output_14_1.png}
    \end{center}
    { \hspace*{\fill} \\}
    
    \section{Blinzel Detektierung}\label{blinzel-detektierung}

Das Blinzeln ist als relativ starker Peak im Signal erkennbar. Das Ziel
ist es nun diese Peaks zu detektieren. Zunächst werden alle lokalen
Maxima bestimmt. Dann wird ein Schwellwert definiert, ab welchen ein
Peak als Blinzeln gezählt wird. Die Größe der gemessenen Spannung und
somit auch das Ausmaß des Ausschlages der Peaks variert von Messung zu
Messung sehr stark. Dies kann von der momentanen Hautleitfähigkeit, aber
auch durch das leicht unterschiedliche Aufbringen der Sensoren
beinflusst werden. Aus diesem Grunde ist es nicht sinnvoll einen fixen,
konkreten Schwellwert zu definieren, sondern ihn bei jeder Session
dynamisch zu berechnen. Hier wird als Schwellwert einfach der größte
gemessene Wert multipliziert mit 0.75 verwendet. Dabei werden zwei
Bedingungen vorausgesetzt. Erstens muss im betrachtetem Intervall
mindestens ein Peak eines Blinzelns vorhanden sein. Ansonst würde ein
viel zu geringer Schwellwert berechnet werden und es würden bereits
kleinere Peaks fälschlicherweise als Blinzeln interpretiert werden.
Zweitens muss das Signal beim nicht Blinzeln um die 0 V sein und der
Ausschlag muss mindestens positiv sein. Dies kann durch Verwendung eines
Tiefpass-Filters garantiert werden.

    \begin{Verbatim}[commandchars=\\\{\}]
{\color{incolor}In [{\color{incolor}27}]:} \PY{k}{def} \PY{n+nf}{detect\PYZus{}peaks}\PY{p}{(}\PY{n}{y}\PY{p}{)}\PY{p}{:}
             \PY{l+s+sd}{\PYZdq{}\PYZdq{}\PYZdq{}Get indices all peaks\PYZdq{}\PYZdq{}\PYZdq{}}
             \PY{c+c1}{\PYZsh{} Peak: value has to be greater than previous and following value}
             \PY{n}{peaks} \PY{o}{=} \PY{p}{[}\PY{n}{i} \PY{k}{for} \PY{n}{i} \PY{o+ow}{in} \PY{n+nb}{range}\PY{p}{(}\PY{l+m+mi}{1}\PY{p}{,} \PY{n+nb}{len}\PY{p}{(}\PY{n}{y}\PY{p}{)} \PY{o}{\PYZhy{}} \PY{l+m+mi}{1}\PY{p}{)} \PY{k}{if} \PY{n}{y}\PY{p}{[}\PY{n}{i}\PY{p}{]} \PY{o}{\PYZgt{}} \PY{n}{y}\PY{p}{[}\PY{n}{i}\PY{o}{\PYZhy{}}\PY{l+m+mi}{1}\PY{p}{]} \PY{o+ow}{and} \PY{n}{y}\PY{p}{[}\PY{n}{i}\PY{p}{]} \PY{o}{\PYZgt{}} \PY{n}{y}\PY{p}{[}\PY{n}{i}\PY{o}{+}\PY{l+m+mi}{1}\PY{p}{]}\PY{p}{]}
             
             \PY{c+c1}{\PYZsh{} Get minimum peak height}
             \PY{n}{mph} \PY{o}{=} \PY{n+nb}{max}\PY{p}{(}\PY{n}{y}\PY{p}{)} \PY{o}{*} \PY{l+m+mf}{0.75} \PY{c+c1}{\PYZsh{} Minimum peak hight}
             
             \PY{c+c1}{\PYZsh{} Get peaks greater than minimum\PYZhy{}peak\PYZhy{}height}
             \PY{n}{peaks} \PY{o}{=} \PY{p}{[}\PY{n}{peak} \PY{k}{for} \PY{n}{peak} \PY{o+ow}{in} \PY{n}{peaks} \PY{k}{if} \PY{n}{y}\PY{p}{[}\PY{n}{peak}\PY{p}{]} \PY{o}{\PYZgt{}}\PY{o}{=} \PY{n}{mph}\PY{p}{]}
             \PY{k}{return} \PY{n}{peaks}
         
         \PY{c+c1}{\PYZsh{} Plot unfiltered Signal with peaks}
         \PY{n}{peaks\PYZus{}indexes} \PY{o}{=} \PY{n}{detect\PYZus{}peaks}\PY{p}{(}\PY{n}{y}\PY{p}{)}
         \PY{k}{for} \PY{n}{peak\PYZus{}index} \PY{o+ow}{in} \PY{n}{peaks\PYZus{}indexes}\PY{p}{:}
             \PY{n}{plt}\PY{o}{.}\PY{n}{plot}\PY{p}{(}\PY{n}{x}\PY{p}{[}\PY{n}{peak\PYZus{}index}\PY{p}{]}\PY{p}{,} \PY{n}{y}\PY{p}{[}\PY{n}{peak\PYZus{}index}\PY{p}{]}\PY{p}{,} \PY{l+s+s1}{\PYZsq{}}\PY{l+s+s1}{bo}\PY{l+s+s1}{\PYZsq{}}\PY{p}{)}
         
         \PY{n}{plt}\PY{o}{.}\PY{n}{plot}\PY{p}{(}\PY{n}{x}\PY{p}{,} \PY{n}{y}\PY{p}{,} \PY{n}{color}\PY{o}{=}\PY{l+s+s1}{\PYZsq{}}\PY{l+s+s1}{red}\PY{l+s+s1}{\PYZsq{}}\PY{p}{)}
         \PY{n}{plt}\PY{o}{.}\PY{n}{title}\PY{p}{(}\PY{l+s+s1}{\PYZsq{}}\PY{l+s+s1}{Unfiltered signal}\PY{l+s+s1}{\PYZsq{}}\PY{p}{)}
         \PY{n}{plt}\PY{o}{.}\PY{n}{xlabel}\PY{p}{(}\PY{l+s+s1}{\PYZsq{}}\PY{l+s+s1}{t (s)}\PY{l+s+s1}{\PYZsq{}}\PY{p}{)}
         \PY{n}{plt}\PY{o}{.}\PY{n}{ylabel}\PY{p}{(}\PY{l+s+s1}{\PYZsq{}}\PY{l+s+s1}{U (μV)}\PY{l+s+s1}{\PYZsq{}}\PY{p}{)}
         \PY{n}{plt}\PY{o}{.}\PY{n}{show}\PY{p}{(}\PY{p}{)}
         \PY{n+nb}{print}\PY{p}{(}\PY{l+s+s1}{\PYZsq{}}\PY{l+s+s1}{Detected blinks: }\PY{l+s+s1}{\PYZsq{}} \PY{o}{+} \PY{n+nb}{str}\PY{p}{(}\PY{p}{(}\PY{n+nb}{len}\PY{p}{(}\PY{n}{peaks\PYZus{}indexes}\PY{p}{)}\PY{p}{)}\PY{p}{)}\PY{p}{)}
         
         \PY{c+c1}{\PYZsh{} Plot filtered Signal with peaks}
         \PY{n}{peaks\PYZus{}indexes} \PY{o}{=} \PY{n}{detect\PYZus{}peaks}\PY{p}{(}\PY{n}{y\PYZus{}filtered}\PY{p}{)}
         \PY{k}{for} \PY{n}{peak\PYZus{}index} \PY{o+ow}{in} \PY{n}{peaks\PYZus{}indexes}\PY{p}{:}
             \PY{n}{plt}\PY{o}{.}\PY{n}{plot}\PY{p}{(}\PY{n}{x}\PY{p}{[}\PY{n}{peak\PYZus{}index}\PY{p}{]}\PY{p}{,} \PY{n}{y\PYZus{}filtered}\PY{p}{[}\PY{n}{peak\PYZus{}index}\PY{p}{]}\PY{p}{,} \PY{l+s+s1}{\PYZsq{}}\PY{l+s+s1}{ro}\PY{l+s+s1}{\PYZsq{}}\PY{p}{)}
         
         \PY{n}{plt}\PY{o}{.}\PY{n}{plot}\PY{p}{(}\PY{n}{x}\PY{p}{,} \PY{n}{y\PYZus{}filtered}\PY{p}{,} \PY{n}{color}\PY{o}{=}\PY{l+s+s1}{\PYZsq{}}\PY{l+s+s1}{blue}\PY{l+s+s1}{\PYZsq{}}\PY{p}{)}
         \PY{n}{plt}\PY{o}{.}\PY{n}{title}\PY{p}{(}\PY{l+s+s1}{\PYZsq{}}\PY{l+s+s1}{Filtered signal}\PY{l+s+s1}{\PYZsq{}}\PY{p}{)}
         \PY{n}{plt}\PY{o}{.}\PY{n}{xlabel}\PY{p}{(}\PY{l+s+s1}{\PYZsq{}}\PY{l+s+s1}{t (s)}\PY{l+s+s1}{\PYZsq{}}\PY{p}{)}
         \PY{n}{plt}\PY{o}{.}\PY{n}{ylabel}\PY{p}{(}\PY{l+s+s1}{\PYZsq{}}\PY{l+s+s1}{U (μV)}\PY{l+s+s1}{\PYZsq{}}\PY{p}{)}
         \PY{n}{plt}\PY{o}{.}\PY{n}{show}\PY{p}{(}\PY{p}{)}
         \PY{n+nb}{print}\PY{p}{(}\PY{l+s+s1}{\PYZsq{}}\PY{l+s+s1}{Detected blinks: }\PY{l+s+s1}{\PYZsq{}} \PY{o}{+} \PY{n+nb}{str}\PY{p}{(}\PY{p}{(}\PY{n+nb}{len}\PY{p}{(}\PY{n}{peaks\PYZus{}indexes}\PY{p}{)}\PY{p}{)}\PY{p}{)}\PY{p}{)}
\end{Verbatim}


    \begin{center}
    \adjustimage{max size={0.9\linewidth}{0.9\paperheight}}{output_16_0.png}
    \end{center}
    { \hspace*{\fill} \\}
    
    \begin{Verbatim}[commandchars=\\\{\}]
Detected blinks: 15

    \end{Verbatim}

    \begin{center}
    \adjustimage{max size={0.9\linewidth}{0.9\paperheight}}{output_16_2.png}
    \end{center}
    { \hspace*{\fill} \\}
    
    \begin{Verbatim}[commandchars=\\\{\}]
Detected blinks: 3

    \end{Verbatim}

    Aus den Ergebnisen kann man erkennen, dass im ersten Fall, bei
Analysierung des ungefilterten Siganls, die Detektierung nicht
funktioniert hat. Es wurden viele lokale Maxima (15) fälschlicherweise
als Blinzeln gewertet. Dies ist auf die Nichterfüllung der zweiten
Bedingung zurückzuführen. Beim Nichtblinzeln beträgt das Signal zwischen
0.2 und 0.3 μV und nicht wie vorausgesetzt um die 0 V. Außerdem schlägt
das Signal beim Blinzeln nur negativ aus. Deshalb wurden trotz der
vielen false-positive Detektierungen die eigentlichen Blizel-Peaks nicht
erkannt. Im zweiten Fall, bei der Analysierung des gefilterten Signals,
wurden das dreifache Blinzeln der Testperson richtig erkannt!


    % Add a bibliography block to the postdoc
    
    
    
    \end{document}
